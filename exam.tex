\documentclass[11pt,a4paper,oneside]{scrartcl}
\usepackage[utf8]{inputenc}
\usepackage[english,russian]{babel}
\usepackage[top=1cm,bottom=1cm,left=1cm,right=1cm]{geometry}
\usepackage{amsmath}
\usepackage{amssymb}
\usepackage{stmaryrd}
\usepackage{cmll}
\usepackage{xcolor}
\usepackage{proof}
\usepackage{comment}
\usepackage{titletoc}
\usepackage{amsthm}
\newtheorem*{thm}{Теорема}
\newtheorem*{dfn}{Определение}
\theoremstyle{definition}
\newtheorem*{exm}{Пример}
\newtheorem*{axm}{Аксиома}
\newtheorem*{lmm}{Лемма}
\newtheorem*{snote}{Замечание}
\setcounter{tocdepth}{1}
\usepackage{hyperref}
\usepackage{tikz}
\usetikzlibrary{hobby,fit,backgrounds,calc,shapes.geometric,patterns}
\hypersetup{
    colorlinks=true,
    linkcolor=blue,
    filecolor=magenta,
    urlcolor=cyan,
}
\usepackage{graphicx} 
\newcommand{\pause}{} 
\newcommand{\mdoubleplus}{\mathbin{+\!\!+}} 

\title{Differential Equations Exam}
\author{Artemiy Maslov}
\date{January 2026}

\begin{document}

    \maketitle

    \newpage

    \tableofcontents

    \newpage

    \section{Общее решение уравнения в полных дифференциалах. Общее решение уравнения с разделяющимися переменными.}

    \begin{dfn}
        Уравнение
        \[
        P(x, y) \, dx + Q(x, y) \, dy = 0 \tag{2.4}
        \]
        называют \textbf{уравнением в полных дифференциалах} в области \( G \), если для него существует \textbf{потенциал}, то есть такая дифференцируемая функция \( u \), что для всех \( x, y \in G \)
        \[
        du = P(x, y) \, dx + Q(x, y) \, dy.
        \]
        \end{dfn}
        
        \begin{thm}(Общее решение УПД)
        Пусть \( G \subset \mathbb{R}^2 \) — область, функция \( u: G \to \mathbb{R} \) дифференцируема, \( u'_x = P \), \( u'_y = Q \). Тогда функция \( y = \varphi(x) \) — решение уравнения (2.4) на промежутке \( E \), если и только если она дифференцируема на \( E \) и при некотором \( C \in \mathbb{R} \) неявно задана уравнением
        \[
        u(x, y) = C.
        \]
        \end{thm}
        
        \begin{proof}\pause
        \textbf{Достаточность.} Дифференцируя равенство \( u(x, \varphi(x)) = C \) по переменной \( x \in E \), находим
        \[
        u'_x(x, \varphi(x)) + u'_y(x, \varphi(x))\varphi'_x \equiv 0.
        \]
        Так как \( u'_x = P \), \( u'_y = Q \), то по определению функция \( \varphi \) является решением уравнения (2.4) на \( E \).
        
        \textbf{Необходимость.} На промежутке \( E \) верно тождество
        \[
        P(x, \varphi(x)) + Q(x, \varphi(x))\varphi'(x) \equiv 0.
        \]
        Левая часть этого равенства совпадает с полной производной функции \( u \) по переменной \( x \). Поэтому
        \[
        \frac{d}{dx}u(x, \varphi(x)) \equiv 0.
        \]
        Следовательно, \( u(x, \varphi(x)) \equiv C \).
        \end{proof}

        Рассмотрим уравнение
\[
P(x, y) \, dx + Q(x, y) \, dy = 0. \tag{3.1}
\]

Предположим, что условие \( P'_y = Q'_x \) нарушается. Это значит, что уравнение (3.1) не является уравнением в полных дифференциалах.

\begin{dfn}
Функция \(\mu: G \to \mathbb{R}\) называется \textbf{интегрирующим множителем} уравнения \(P(x, y) \, dx + Q(x, y) \, dy = 0\) в области \(G\), если \(\mu(x, y) \neq 0\) для любой точки \((x, y) \in G\) и уравнение
\[
\mu(x, y)P(x, y) \, dx + \mu(x, y)Q(x, y) \, dy = 0 \tag{3.2}
\]
является уравнением в полных дифференциалах в области \(G\).
\end{dfn}



Пусть \(p_2(x) \neq 0\) при \(x \in (a, b)\), \(q_1(y) \neq 0\) при \(y \in (c, d)\). Тогда функция
\[
\mu(x, y) = \frac{1}{p_2(x)q_1(y)}
\]
является интегрирующим множителем для уравнений
\[
p_1(x)q_1(y) \, dx + p_2(x)q_2(y) \, dy = 0 \tag{3.3}
\]
в области \((a, b) \times (c, d)\). Действительно, умножая обе части уравнения (3.3) на \(\mu(x, y)\), получаем уравнение с разделёнными переменными
\[
\frac{p_1(x)}{p_2(x)} \, dx + \frac{q_2(y)}{q_1(y)} \, dy = 0.
\]

\begin{dfn}
Уравнение (3.3) называют \textbf{уравнением с разделяющимися переменными}.
\end{dfn}

\section{Общее решение линейного уравнения 1-го порядка. Общее решение линейного однородного уравнения 1-го порядка.}

\begin{dfn}
    Дифференциальное уравнение
    \[
    y' = p(x)y + q(x) \tag{3.5}
    \]
    называется \textbf{линейным уравнением} первого порядка.
    \end{dfn}
    
    \begin{dfn}
    Уравнение (3.5) — \textbf{линейное однородное}, если \(q = 0\), иначе — \textbf{линейное неоднородное}.
    \end{dfn}

\begin{thm}(Общее решение ЛУ 1-го порядка)
    Пусть \( E = \langle a, b \rangle \), \( p, q \in C(E) \), \( \mu = e^{-\int p} \). Тогда общее решение уравнения (3.5) имеет вид
    \[
    y = \frac{C + \int q\mu}{\mu}, \quad C \in \mathbb{R}, \quad \text{dom } y = E.
    \tag{3.8}
    \]
\end{thm}
    
    \begin{proof}
    Умножая (3.5) на \( e^{-\int p} \), получаем
    \[
    y' e^{-\int p} - py e^{-\int p} = qe^{-\int p}.
    \]
    Левая часть — производная произведения \( e^{-\int p} \) и \( y \). Тогда
    \[
    \left( ye^{-\int p} \right)' = \int qe^{-\int p}.
    \]
    Следовательно,
    \[
    ye^{-\int p} = C + \int qe^{-\int p}.
    \]
    При умножении данного равенства на \( e^{\int p} \) приходим к формуле (3.8). \qedhere
    \end{proof}
    
    \begin{thm}(Общее решение ЛОУ 1-го порядка)
    Пусть \( E = \langle a, b \rangle \), \( p \in C(E) \). Тогда общее решение уравнения
    \[
    y' = p(x)y
    
    \]
    имеет вид
    \[
    y = Ce^{\int p}, \quad C \in \mathbb{R}, \quad x \in E.
    \]
    \end{thm}
    
    \begin{proof}
    Достаточно положить \( q = 0 \) в формуле (3.8). \qedhere
    \end{proof}

    \section{Нормальная система, равносильная каноническому уравнению.}

    \begin{dfn}
        Зададим отображение \(\Lambda_n\) формулой
        \[
        \Lambda_n \varphi = (\varphi, \varphi', \ldots, \varphi^{(n-1)})^T.
        \]
        Индекс \(n\) будем опускать, если его значение ясно из контекста.
        \end{dfn}
        
        \begin{lmm}
        Отображение \(\Lambda_n\) — биекция между решениями уравнения
        \[
        y^{(n)} = f(t, y, y', \ldots, y^{(n-1)}) \tag{5.11}
        \]
        и решениями системы
        \[
        r' = 
        \begin{bmatrix}
        r_2 \\
        r_3 \\
        \vdots \\
        r_n \\
        f(t, r)
        \end{bmatrix}. \tag{5.12}
        \]
        \end{lmm}
        
        \begin{proof}
        Пусть \(y\) — решение уравнения (5.11). Пусть \(r = \Lambda_n y\), то есть
        \[
        r_1 = y, \quad r_2 = y', \quad r_3 = y'', \quad \ldots, \quad r_{n-1} = y^{(n-2)}, \quad r_n = y^{(n-1)}.
        \]
        Дифференцируя каждое равенство, имеем
        \begin{align*}
        r'_1 &= y' = r_2, \\
        r'_2 &= y'' = r_3, \\
        &\ldots \\
        r'_{n-1} &= y^{(n-1)} = r_n, \\
        r'_n &= y^{(n)} = f(t, y, y', \ldots, y^{(n-1)}) = f(t, r_1, r_2, \ldots, r_n).
        \end{align*}
        Следовательно, вектор-функция \(\Lambda_n y\) — решение системы (5.12).
        
        Если \(y_1, y_2\) — разные функции, то \(\Lambda_n y_1, \Lambda_n y_2\) — разные вектор-функции, так как они отличаются хотя бы первыми компонентами. Следовательно, \(\Lambda_n\) — инъекция.
        
        Пусть \(r\) — решение системы (5.12). Дифференцируя первое уравнение системы и принимая во внимание второе, получаем \(r''_1 = r_3\). Дифференцируя полученное равенство и принимая во внимание третье уравнение системы, имеем \(r^{(3)}_1 = r_4\). Продолжая аналогично, находим \(r^{(n-1)}_1 = r_n\). Дифференцируя это равенство и принимая во внимание последнее уравнение системы, получаем \(r_1^{(n)} = f(t, r)\). Учитывая все найденные соотношения, имеем \(r = \Lambda_n r_1\). Это означает, что \(r_1\) — прообраз вектор-функции \(r\) в множестве решений уравнения (5.11) при отображении \(\Lambda_n\). Следовательно, \(\Lambda_n\) — сюръекция.
        \end{proof}

    \section{Достаточное условие локальной липшицевости.}

    \begin{lmm}
        Пусть \( f \in C([a, b] \to \mathbb{R}^n) \). Тогда
        \[
        \left| \int_a^b f(\tau) \, d\tau \right| \leq \int_a^b |f(\tau)| \, d\tau.
        \]
        \end{lmm}
        
        \begin{proof}
        Принимая во внимание определение нормы, имеем
        \begin{align*}
        \left| \int_a^b f(\tau) \, d\tau \right| &= \max_i \left| \int_a^b f_i(\tau) \, d\tau \right| \\
        &\leq \max_i \int_a^b |f_i(\tau)| \, d\tau \\
        &\leq \max_i \int_a^b \max_j |f_j(\tau)| \, d\tau \\
        &= \max_i \int_a^b |f(\tau)| \, d\tau = \int_a^b |f(\tau)| \, d\tau.
        \end{align*}
        \end{proof}
        
        \begin{lmm}
        Пусть \( A \in \operatorname{Mat}_{m \times n}(\mathbb{R}) \), \( B \in \operatorname{Mat}_{n \times l}(\mathbb{R}) \). Тогда
        \[
        |AB| \leq n|A||B|.
        \]
        \end{lmm}
        
        \begin{proof}
        Пусть \( AB = C \). Тогда
        \[
        |C_j^i| = \left| \sum_{k=1}^n A_k^i B_k^j \right| \leq \sum_{k=1}^n |A_k^i||B_k^j| \leq \sum_{k=1}^n |A||B| = n|A||B|.
        \]
        \end{proof}

    \begin{dfn}
        Функция \( f: G \subset \mathbb{R}^{n+1}_{t,r} \to \mathbb{R}^n \) удовлетворяет условию Липшица по \(r\) (равномерно по \(t\)) на множестве \( G \), если найдётся \( L \in \mathbb{R} \), такое что для любых точек \( (t, r^1), (t, r^2) \in G \) справедливо неравенство
        \[
        |f(t, r^2) - f(t, r^1)| \leq L|r^2 - r^1|.
        \]
        Обозначение: \( f \in \operatorname{Lip}_r G \).
    \end{dfn}
        
        \begin{dfn}
        Функция \( f: G \subset \mathbb{R}^{n+1}_{t,r} \to \mathbb{R}^n \) удовлетворяет условию Липшица по \(r\) локально на множестве \( G \), если для любой точки \( x \in G \) можно указать её окрестность \( U(x) \), такую что \( f \in \operatorname{Lip}_r(U(x) \cap G) \). Обозначение: \( f \in \operatorname{Lip}_{r,\text{loc}} G \).
        \end{dfn}
        
        \begin{lmm}(Достаточное условие локальной липшицевости)
        Пусть \( G \subset \mathbb{R}^{n+1}_{t,r} \) — область, \( f \in C(G \to \mathbb{R}^n), \, f'_r \in \operatorname{Mat}_n(C(G)) \). Тогда \( f \in \operatorname{Lip}_{r,\text{loc}} G \).
        \end{lmm}
        
        \begin{proof}
        Возьмём произвольную точку из области \( G \) и построим открытый шар \( B \subset G \) с центром в этой точке.
        
        Пусть \( (t, r^1), (t, r^2) \in B \). В силу выпуклости шара \( B \) будет \( (t, r^1 + s(r^2 - r^1)) \in B \) при \( s \in [0, 1] \). Положим
        \[
        g(s) = f(t, r^1 + s(r^2 - r^1)).
        \]
        Тогда
        \begin{align*}
        f(t, r^2) - f(t, r^1) &= g(1) - g(0) = \int_0^1 g'(s) \, ds \\
        &= \int_0^1 f'_r \cdot r'_s \, ds \\
        &= \int_0^1 f'_r(t, r^1 + s(r^2 - r^1)) \cdot (r^2 - r^1) \, ds.
        \end{align*}
        Принимая во внимание неравенство для интегралов и неравенство для матриц, получаем
        \begin{align*}
        |f(t, r^2) - f(t, r^1)| &\leq \int_0^1 n |f_r'(t, r^1 + s(r^2 - r^1))| |r^2 - r^1| \, ds \\
        &\leq n \sup_{x \in B} |f_r'(x)| \cdot |r^2 - r^1|.
        \end{align*}
        Следовательно, \( f \in \operatorname{Lip}_r B \). По определению будет \( f \in \operatorname{Lip}_{r,\text{loc}} G \).
        \end{proof}

    \section{Достаточное условие глобальной липшицевости}

    \begin{lmm}
        Пусть область \( G \subset \mathbb{R}^{n+1}_{t,r} \), \( f \in C(G \to \mathbb{R}^n) \cap \operatorname{Lip}_{r,\text{loc}} G \), компакт \( K \subset G \). Тогда \( f \in \operatorname{Lip}_r K \).
        \end{lmm}
        
        \begin{proof}
        Докажем методом от противного. Пусть \( f \notin \operatorname{Lip}_r K \). Тогда для любого \( N \in \mathbb{N} \) найдётся пара точек \( (t_N, r^N), (t_N, \tilde{r}^N) \in K \), для которых верно неравенство
        \[
        |f(t_N, r^N) - f(t_N, \tilde{r}^N)| > N|r^N - \tilde{r}^N|. \tag{6.1}
        \]
        
        Поскольку \( K \) — компакт, то из последовательности \( \{(t_N, r^N)\} \) можно выбрать подпоследовательность с номерами \( \{N_k\} \), сходящуюся к некоторой точке \( (t, r) \in K \). Затем из последовательности \( \{(t_{N_k}, \tilde{r}_{N_k})\} \) выберем подпоследовательность с номерами \( \{N_{k_l}\} \), сходящуюся к \( (t, \tilde{r}) \). Пусть \( \nu = \{N_{k_l} \mid l \in \mathbb{N}\} \).
        
        Возможны два случая: \( r = \tilde{r} \) и \( r \neq \tilde{r} \). Рассмотрим сначала первый.
        
        По условию \( f \in \operatorname{Lip}_{r,\text{loc}} G \), значит, найдётся окрестность \( U \) точки \( (t, r) \), в которой \( f \in \operatorname{Lip}_r U \), то есть существует постоянная \( L \), для которой
        \[
        |f(\tau, \rho) - f(\tau, \tilde{\rho})| \leq L|\rho - \tilde{\rho}|
        \]
        при любых \( (\tau, \rho), (\tau, \tilde{\rho}) \in U \). Выберем номер \( N \in \nu \) так, чтобы \( N > L \) и \( (t_N, r^N), (t_N, \tilde{r}^N) \in U \), и положим \( \tau = t_N, \rho = r^N, \tilde{\rho} = \tilde{r}^N \). Тогда из неравенства (6.1) следует
        \[
        |f(\tau, \rho) - f(\tau, \tilde{\rho})| > N|\rho - \tilde{\rho}| \geq L|\rho - \tilde{\rho}|,
        \]
        что противоречит предыдущему неравенству.
        
        Пусть теперь \( r \neq \tilde{r} \). В неравенстве (6.1) перейдём к пределу при \( \nu \ni N \to \infty \).  
        В силу непрерывности функции \( f \) получаем
        \[
        |f(t, r) - f(t, \tilde{r})| \geq \infty,
        \]
        что неверно.
        \end{proof}

    \section{Лемма о равносильном интегральном уравнении.}

    \begin{dfn}
        Пусть \( f: G \subset \mathbb{R}^{n+1} \to \mathbb{R}^n \). Функция \( \varphi: E \to \mathbb{R}^n \) — \textbf{решение на \( E \)} интегрального уравнения
        \[
        r(t) = r^0 + \int_{t_0}^t f(\tau, r(\tau)) \, d\tau,
        \]
        если \( E = \langle a, b \rangle \) и \( \varphi(t) \equiv r^0 + \int_{t_0}^t f(\tau, \varphi(\tau)) \, d\tau \) на \( E \), где интеграл понимается в смысле Римана.
        \end{dfn}
        
        \begin{lmm}
        Пусть \( E = \langle a, b \rangle \), \( t_0 \in E \), \( G \) — область в \( \mathbb{R}^{n+1} \), \( (t_0, r^0) \in G \), \( f \in C(G \to \mathbb{R}^n) \). Тогда \( \varphi \) — решение на \( E \) задачи Коши
        \[
        r' = f(t, r), \quad r(t_0) = r^0, \tag{7.1}
        \]
        если и только если \( \varphi \) — \textbf{решение на \( E \)} уравнения
        \[
        r(t) = r^0 + \int_{t_0}^t f(\tau, r(\tau)) \, d\tau. \tag{7.2}
        \]
        \end{lmm}
        
        \begin{proof}
        Пусть \( \varphi \) — решение (7.1) на \( E \). Интегрируя равенство \( \varphi'(\tau) = f(\tau, \varphi(\tau)) \) от \( t_0 \) до \( t \in E \), обе части которого — непрерывные функции, имеем
        \[
        \varphi(t) - \varphi(t_0) = \int_{t_0}^t f(\tau, \varphi(\tau)) \, d\tau.
        \]
        Поскольку \( \varphi(t_0) = r^0 \), то функция \( \varphi \) — решение уравнения (7.2) по определению.
        
        Докажем обратное. Пусть \( \varphi \) — решение (7.2) на \( E \). Тогда из равенства
        \[
        \varphi(t) = r^0 + \int_{t_0}^t f(\tau, \varphi(\tau)) \, d\tau \tag{7.3}
        \]
        следует, что \( \varphi \in C(E) \). Отсюда и из (7.3) вытекает дифференцируемость \( \varphi \). Дифференцируя (7.3) по \( t \), получаем: \( \varphi'(t) \equiv f(t, \varphi(t)) \). Кроме того, из (7.3) имеем \( \varphi(t_0) = r^0 \). Таким образом, \( \varphi \) — решение задачи (7.1) по определению.
        \end{proof}

    \section{Лемма о гладкой стыковке.}

    \begin{lmm}
        Пусть область \( G \subset \mathbb{R}^{n+1}_{t,r} \), \( f \in C(G \to \mathbb{R}^n) \), \( (t_0, r^0) \in G \), уравнение \( r' = f(t, r) \) имеет решения: \( \varphi_- \) на \((a, t_0)\), \( \varphi_+ \) на \((t_0, b)\). Кроме того, \( \varphi_-(t_0-) = \varphi_+(t_0+) = r^0 \). Тогда функция
        \[
        \varphi(t) =
        \begin{cases} 
        \varphi_-(t), & \text{если } t \in (a, t_0), \\
        r^0, & \text{если } t = t_0, \\
        \varphi_+(t), & \text{если } t \in (t_0, b)
        \end{cases}
        \]
        является решением этого уравнения на \((a, b)\).
        \end{lmm}
        
        \begin{proof}
        Пусть \( t, t_- \in (a, t_0) \). По лемме о равносильном интегральном уравнении
        \[
        \varphi_-(t) = \varphi_-(t_-) + \int_{t_-}^t f(\tau, \varphi_-(t)) \, d\tau.
        \]
        Переходя в этом равенстве к пределу при \( t_- \to t_0^- \) и замечая, что \( \varphi_- = \varphi \) для точек из отрезка \( t, t_- \), получаем
        \[
        \varphi(t) = r^0 + \int_{t_0}^t f(\tau, \varphi(\tau)) \, d\tau. \tag{7.4}
        \]
        Поступая аналогично для точек \( t, t_+ \in (t_0, b) \), при \( t_+ \to t_0^+ \) приходим к равенству (7.4).
        
        Таким образом, равенство (7.4) выполнено для всех \( t \in (a, b) \). Остаётся применить лемму о равносильном интегральном уравнении, из которой следует, что функция \( \varphi \) является решением уравнения на \( (a, b) \).
        \end{proof}

    \section{Лемма Гронуолла.}

    \begin{lmm}
        Пусть \(D = \langle a, b \rangle\), \(\varphi \in C(D)\), \(t_0 \in D\), \(\lambda, \mu \ge 0\), при любом \(t \in D\) верно двойное неравенство
        \[
        0 \le \varphi(t) \le \lambda + \mu \left| \int_{t_0}^t \varphi(\tau) \, d\tau \right|.
        \]
        Тогда для любого \(t \in D\)
        \[
        \varphi(t) \le \lambda e^{\mu |t - t_0|}.
        \]
        \end{lmm}
        
        \begin{proof}
        Рассмотрим случай \(t \ge t_0\) (при \(t < t_0\) доказательство аналогично). Положим
        \[
        v(t) = \lambda + \mu \int_{t_0}^t \varphi(\tau) \, d\tau.
        \]
        Из данного в условии неравенства имеем \(v'(t) = \mu \varphi(t) \le \mu v(t)\). Умножая полученное неравенство на \(e^{-\mu t}\), находим
        \[
        v' e^{-\mu t} - \mu v e^{-\mu t} \le 0,
        \]
        то есть
        \[
        (ve^{-\mu t})' \le 0.
        \]
        Следовательно, функция \(ve^{-\mu t}\) убывает при \(t \ge t_0\). Поэтому
        \[
        v(t) e^{-\mu t} \le v(t_0) e^{-\mu t_0}.
        \]
        Отсюда
        \[
        \varphi(t) \le v(t) \le v(t_0) e^{\mu (t - t_0)} = \lambda e^{\mu (t - t_0)}.
        \]
        \end{proof}

    \section{Теорема Пикара (доказательство существования решения).}

    \begin{dfn}
        Пусть область \( G \subset \mathbb{R}^{n+1}_{t,r}, (t_0, r^0) \in G \),
        
        \[
        \Pi := \{ (t, r) \in \mathbb{R}^{n+1} \mid |t - t_0| \leq a, |r - r^0| \leq b \},
        \]
        
        где числа \( a, b > 0 \) таковы, что \(\Pi \subset G\). Положим \( M = \max_{(t, r) \in \Pi} |f(t, r)|, h = \min\{a, \frac{b}{M}\}\) (если \( M = 0 \), то \( h := a \)). Отрезок \([t_0 - h, t_0 + h]\) называется \textbf{отрезком Пеано}, соответствующим точке \((t_0, r^0)\).
        \end{dfn}
        
        \begin{figure}[h]
        \centering
        \includegraphics{pikar.png}
        \caption{Отрезок Пеано \([t_0 - h, t_0 + h]\)}
        \label{fig:peano}
        \end{figure}
        
        \begin{thm}(Пикар, существование и единственность решения ЗК)
        Пусть область \( G \subset \mathbb{R}^{n+1}_{t,r}, f \in C(G \to \mathbb{R}^n) \cap \operatorname{Lip}_{r,\text{loc}}, (t_0, r_0) \in G \). Тогда
        
        \begin{enumerate}
            \item[(i)] на отрезке Пеано существует решение задачи
            \[
            r' = f(t, r), \quad r(t_0) = r_0; \tag{7.5}
            \]
            \item[(ii)] если \(\psi^1\) и \(\psi^2\) — решения (7.5), то \(\psi^1 \equiv \psi^2\) на \(\operatorname{dom} \psi^1 \cap \operatorname{dom} \psi^2\).
        \end{enumerate}
        \end{thm}
        \textbf{Доказательство существования}
        
        Будем считать, что \( t_0 = 0, r^0 = 0 \) (в противном случае перенесём начало координат в точку \((t_0, r^0)\)). Достаточно установить существование решения на отрезке \([0, h]\) — правой половине отрезка Пеано (рассуждения на \([-h, 0]\) аналогичны, а на всём отрезке решение получается применением леммы о гладкой стыковке).
        
        Пусть
        \[
        \Pi := \{ (t, r) \in \mathbb{R}^{n+1} \mid |t| \leq a, |r| \leq b \} \subset G, \quad M := \max_{\Pi} |f|, \quad h = \min\{a, b/M\}.
        \]
        
        На отрезке \([0, h]\) зададим последовательность функций
        \[
        \varphi^0(t) = 0, \quad \varphi^{k+1}(t) = \int_0^t f(\tau, \varphi^k(\tau)) \, d\tau.
        \]
        
        Дальнейшую часть доказательства разобьём на этапы:
        
        \begin{enumerate}
            \item Чтобы построить функцию \(\varphi^{k+1}\) должно быть \((t, \varphi^k(t)) \in G\) при всех \(t \in [0, h]\). Докажем более сильное утверждение: \((t, \varphi^k(t)) \in \Pi\) при всех \(t \in [0, h]\).
            \item Покажем, что последовательность \((\varphi^k)\) равномерно на \([0, h]\) сходится к некоторой функции \(\varphi\).
            \item Установим, что \(\varphi\) — решение интегрального уравнения, равносильного задаче (7.5). Тогда останется применить лемму о равносильном интегральном уравнении для завершения доказательства пункта (i).
        \end{enumerate}
        
        \textbf{Этап 1.} При \(k = 0\), очевидно, \((t, \varphi^0(t)) \in \Pi\). Пусть это верно при некотором \(k \in \mathbb{Z}_+\). Тогда функция \(\varphi^{k+1}\) определена на \([0, h]\) и
        \[
        |\varphi^{k+1}(t)| \leq \int_0^t |f(\tau, \varphi^k(\tau))| \, d\tau \leq Mt \leq Mh \leq M \frac{b}{M} = b,
        \]
        что влечёт включение \((t, \varphi^{k+1}(t)) \in \Pi\) при всех \(t \in [0, h]\).
        
        \textbf{Этап 2.} Воспользуемся критерием Коши: установим, что для любого \(\varepsilon > 0\) найдётся \(N \in \mathbb{N}\), такое что при всех \(m \geq N\), всех \(k \in \mathbb{N}\) и всех \(t \in [0, h]\)
        \[
        |\varphi^{m+k}(t) - \varphi^m(t)| \leq \varepsilon.
        \]
        
        По достаточному условию глобальной липшицевости \(f \in \operatorname{Lip}_r\), \(\Pi\) с некоторой константой Липшица \(L\). Индукцией по \(m\) докажем неравенство
        \[
        |\varphi^{m+k}(t) - \varphi^m(t)| \leq \frac{ML^m t^{m+1}}{(m+1)!}. \tag{7.6}
        \]
        
        При \(m = 0\) утверждение верно, так как
        \[
        |\varphi^k(t) - \varphi^0(t)| \leq \int_0^t |f(\tau, \varphi^{k-1}(\tau))| \, d\tau \leq Mt.
        \]
        
        Допуская его справедливость при некотором \(m\), имеем
        \begin{align*}
        |\varphi^{m+1+k}(t) - \varphi^{m+1}(t)| &\leq \int_0^t |f(\tau, \varphi^{m+k}(\tau)) - f(\tau, \varphi^m(\tau))| \, d\tau \\
        &\leq \int_0^t L |\varphi^{m+k}(\tau) - \varphi^m(\tau)| \, d\tau \\
        &\leq \int_0^t L \frac{ML^m \tau^{m+1}}{(m+1)!} \, d\tau = \frac{ML^{m+1} \tau^{m+2}}{(m+2)!}.
        \end{align*}
        
        Из (7.6) вытекает, что при любом \(t \in [0, h]\)
        \[
        |\varphi^{m+k}(t) - \varphi^m(t)| \leq \frac{ML^mh^{m+1}}{(m+1)!}. \tag{7.7}
        \]
        
        Выражение в правой части не зависит от \(t\) и \(k\) и стремится к нулю при \(m \to \infty\), поскольку является общим членом ряда Тейлора для экспоненты. Значит, последовательность \((\varphi^m)\) удовлетворяет критерию Коши. Обозначим через \(\varphi\) её предел на \([0, h]\).
        
        \textbf{Этап 3.} Переходя к пределу при \(m \to \infty\) в равенстве
        \[
        \varphi^{m+1}(t) = \int_0^t f(\tau, \varphi^m(\tau)) \, d\tau,
        \]
        получаем
        \[
        \varphi(t) = \lim_{m \to +\infty} \int_0^t f(\tau, \varphi^m(\tau)) \, d\tau. \tag{7.8}
        \]
        
        На этапе 1 было установлено, что \((t, \varphi^m(t)) \in \Pi\) при всех \(t \in [0, h]\). Тогда при \(m \to +\infty\) будет \((t, \varphi(t)) \in \Pi\) при всех таких \(t\). Следовательно,
        \[
        |f(\tau, \varphi^m(\tau)) - f(\tau, \varphi(\tau))| \leq L |\varphi^m(\tau) - \varphi(\tau)|.
        \]
        
        Учитывая равномерную сходимость \(\varphi^m\), из данного неравенства следует, что \(f(t, \varphi^m(t)) \to f(t, \varphi(t))\) при \(m \to +\infty\) равномерно на \([0, h]\). Это позволяет внести знак предела под интеграл в (7.8). После этого по лемме о равносильном уравнении заключаем, что \(\varphi\) — решение задачи (7.5) на \([0, h]\). \square
        

    \section{Теорема Пикара (доказательство единственности решения).}

    \textbf{Доказательство единственности}. Пусть \(\psi^1\) и \(\psi^2\) — решения (7.5), \(E = \operatorname{dom} \psi^1 \cap \operatorname{dom} \psi^2\). По лемме о равносильном интегральном уравнении
        \[
        \psi^k(t) = \int_0^t f(\tau, \psi^k(\tau)) \, d\tau, \quad t \in E, \quad k \in \{1, 2\},
        \]
        поэтому
        \[
        |\psi^1(t) - \psi^2(t)| \leq \int_0^t |f(\tau, \psi^1(\tau)) - f(\tau, \psi^2(\tau))| \, d\tau.
        \]
        
        Рассмотрим произвольный отрезок \([\alpha, \beta] \subset E\), содержащий ноль. Графики функций \(\psi^1\) и \(\psi^2\) на \([\alpha, \beta]\) — компактные множества. По достаточному условию глобальной липшицевости найдётся постоянная \(\tilde{L}\), такая что
        \[
        |f(\tau, \psi^1(\tau)) - f(\tau, \psi^2(\tau))| \leq \tilde{L} |\psi^1(\tau) - \psi^2(\tau)|
        \]
        при всех \(\tau \in [\alpha, \beta]\). Следовательно,
        \[
        |\psi^1(t) - \psi^2(t)| \leq \tilde{L} \int_0^t |\psi^1(\tau) - \psi^2(\tau)| \, d\tau.
        \]
        
        По лемме Гронуолла будет \(|\psi^1(t) - \psi^2(t)| = 0\) на \([\alpha, \beta]\), то есть \(\psi^1\) и \(\psi^2\) совпадают на \([\alpha, \beta]\). Поскольку отрезок \([\alpha, \beta]\) выбирался произвольно из \(E\), то функции \(\psi^1\) и \(\psi^2\) совпадают и на всём промежутке \(E\). \square

    \section{Критерий локальной продолжимости.}

    \begin{dfn}
        Решение $\varphi$ уравнения $r' = f(t, r)$ \textbf{продолжимо}, если существует решение $\psi$ того же уравнения, такое что $\operatorname{dom} \varphi \subsetneq \operatorname{dom} \psi$ и $\psi|_{\operatorname{dom} \varphi} = \varphi$.  
        Решение $\psi$ называют \textbf{продолжением решения} $\varphi$.
        \end{dfn}
        
        \begin{dfn}
        Если для решения $\varphi$ уравнения $r' = f(t, r)$ не существует продолжения, то функция $\varphi$ — \textbf{максимальное решение} этого уравнения.
        \end{dfn}
      
        \begin{thm}
        Пусть область $G \subset \mathbb{R}^{n+1}_{t,r}$, $f \in C(G \to \mathbb{R}^n) \cap \operatorname{Lip}_{r,\text{loc}}$, $\varphi$ — решение уравнения $r' = f(t, r)$ на промежутке $[a, b]$. Тогда решение $\varphi$ продолжимо на отрезок $[a, c]$ при некотором $c > b$, если и только если $(b, \varphi(b-)) \in G$.
        \end{thm}
        
        \begin{proof}
        \textbf{Необходимость.} Пусть $\psi$ — продолжение на $[a, c]$ решения $\varphi$. Тогда в силу непрерывности функции $\psi$
        \[
        \varphi(b-) = \psi(b-) = \psi(b).
        \]
        Поскольку $b \in [a, c]$, то из определения решения следует $(b, \psi(b)) \in G$.  
        
        \textbf{Достаточность.} По теореме Пикара 7.2.1 существует решение $\chi$ задачи
        \[
        r' = f(t, r), \quad r(b) = \varphi(b-)
        \]
        на некотором отрезке $[b, b+h]$. Положим (рис. 8.1)
        \[
        \psi(t) := 
        \begin{cases} 
        \varphi(t), & t \in [a, b), \\
        \chi(t), & t \in [b, b+h].
        \end{cases}
        \]
        По лемме о гладкой стыковке функция $\psi$ — решение уравнения $r' = f(t, r)$ на $[a, b+h]$. Тогда $\psi$ — продолжение решения $\varphi$ на $[a, c]$, где $c = b + h$.
        \end{proof}

    \section{Теорема существования и единственности максимального решения задачи Коши.}

    \begin{thm}
        Пусть область \( G \subset \mathbb{R}^{n+1}_{t,r}, f \in C(G \to \mathbb{R}^n) \cap \text{Lip}_{r,loc}, (t_0, r^0) \in G \). Тогда
        
        (i) существует максимальное решение \(\psi\) задачи Коши  
        \[
        r' = f(t, r), \quad r(t_0) = r^0;
        \tag{8.1}
        \]
        
        (ii) любое решение задачи (8.1) — сужение решения \(\psi\).
    \end{thm}
        
        \begin{proof}
        (i) Пусть \(S\) — множество всевозможных решений ЗК, определённых на произвольных промежутках. По теореме Пикара это множество не пусто. Обозначим через \(a_\varphi\) и \(b_\varphi\) левый и правый конец промежутка \(\operatorname{dom} \varphi\). Положим  
        \[
        a = \inf_{\varphi \in S} a_\varphi, \quad b = \sup_{\varphi \in S} b_\varphi.
        \]
        
        Определим на \((a, b)\) функцию \(\psi\) следующим образом. Пусть \(t \in [t_0, b)\). Возьмём произвольное решение \(\varphi\), для которого \(t < b_\varphi\) (такое решение найдётся в силу определения числа \(b\)). Положим \(\psi(t) = \varphi(t)\).
        
        Если найдётся ещё одно решение \(\varphi_1\), такое что \(t < b_{\varphi_1}\), то \(\varphi(t) = \varphi_1(t)\) по теореме Пикара. Тем самым, в точке \(t\) функция \(\psi\) определена однозначно.
        
        Из определения функции \(\psi\) следует, что \(\psi \equiv \varphi\) на \([t_0, b_\varphi)\).
        
        Тогда \(\psi\) — решение на \([t_0, b_\varphi]\) ЗК. Следовательно, \(\psi'(t) = f(t, \psi(t))\). Так как \(t\) выбиралось произвольно из \([t_0, b)\), то последнее равенство верно на \([t_0, b)\). То есть \(\psi\) — решение на \([t_0, b)\).
        
        Аналогично поступаем при \(t \in (a, t_0]\). По лемме о гладкой стыковке функция \(\psi\) будет решением на \((a, b)\).
        
        Если решение \(\psi\) продолжимо на промежуток \((a, b]\), то по критерию локальной продолжимости оно продолжимо и на промежуток \((a, c]\) при некотором \(c > b\). Но это противоречит
        определению числа \(b\). Аналогично для точки \(a\). Таким образом, \(\psi\) — максимальное решение.
        
        (ii) Пусть \(\varphi \in S\). По теореме Пикара будет \(\psi \equiv \varphi\) на \(\operatorname{dom} \psi \cap \operatorname{dom} \varphi = (a_\varphi, b_\varphi)\). Так как \((a_\varphi, b_\varphi) \subset (a, b)\), то \(\varphi\) — сужение функции \(\psi\).
        \end{proof}
    
        \section{Теорема о выходе интегральной кривой за пределы компакта.}
        \begin{thm}
            Пусть \( n \in \mathbb{N} \), область \( G \subset \mathbb{R}^{n+1}_r \), \( f \in C(G \to \mathbb{R}^n) \cap \text{Lip}_{loc} \), \( \varphi \) — максимальное решение на \((a, b)\) уравнения \( r' = f(t, r) \), \( K \subset G \) — компакт. Тогда найдётся \( \Delta > 0 \), такое что \((t, \varphi(t)) \notin K\) при всех \( t \in (a, a + \Delta) \cup (b - \Delta, b) \).
            \end{thm}
            
            \begin{proof}
            Заметим, что расстояние \( \rho = \rho(K, \partial G) \) от компакта \( K \) до границы \( \partial G \) области \( G \) положительно (иначе можно было бы построить последовательность точек из \( K \), сходящейся к точке на границе, но \( \partial G \cap K = \emptyset \)). Если \( \rho < +\infty \), положим \( c := \rho/2 \), иначе пусть \( c := 1 \).
            
            Вокруг каждой точки \( p' \in K \) построим внутри \( G \) параллелепипед
            
            \[
            \Pi(p') = \{ p \in \mathbb{R}^{n+1} \mid |p - p'| \leq c \}
            \]
            
            и рассмотрим множество
            
            \[
            K_c := \bigcup_{p' \in K} \Pi(p').
            \]
            
            Поскольку \( K \) — компакт, то норма каждого элемента из \( K \) ограничена некоторым числом \( d \). Если \( p \) — произвольная точка из \( K_c \), то для некоторой точки \( p' \in K \) будет \( p \in \Pi(p') \), поэтому
            
            \[
            |(t, r)| \leq |(t, r) - (t', r')| + |(t', r')| \leq c + d.
            \]
            
            Значит, множество \( K_c \) ограничено.
            
            Докажем его замкнутость. Рассмотрим последовательность \( (p_m) \) точек из \( K_c \), сходящуюся к \( p \in \mathbb{R}^{n+1} \). Для каждой такой точки найдётся параллелепипед \( \Pi(p'_m) \), которому она принадлежит. Раз \( K \) — компакт, то существует подпоследовательность \( (p'_m) \), сходящаяся к некоторой точке \( p' \in K \). Переходя к пределу в неравенствах
            
            \[
            |p_{m_k} - p'_{m_k}| \leq c,
            \]
            
            находим \( |p - p'| \leq c \). Следовательно, \( p \in K_c \).
            
            Таким образом, \( K_c \) — компакт, и функция \( |f| \) достигает на нём максимального значения
            
            \[
            M := \max_{p \in K_c} |f(p)|.
            \]
            
            Теперь предположим, что утверждение теоремы неверно. Пусть \( \Delta = h/2 \), где \( h = \min\{c, c/M\} \). Тогда при некотором \( t_0 \in (b - h/2, b) \) будет \((t_0, \varphi(t_0)) \in K\).
            
            Рассмотрим задачу Коши \( r' = f(t, r) \) с начальными данными \( r(t_0) = \varphi(t_0) \). По теореме Пикара она имеет решение \( \psi \) на отрезке \([t_0, t_0 + h]\). Пусть
            
            \[
            \tilde{\varphi}(t) =
            \begin{cases} 
            \varphi(t), & \text{если } t \in (a, t_0), \\
            \psi(t), & \text{если } t \in [t_0, t_0 + h].
            \end{cases}
            \]
            
            По лемме о гладкой стыковке решений \( \tilde{\varphi} \) — решение уравнения \( r' = f(t, r) \) на \((a, t_0 + h)\). Функция \( \tilde{\varphi} \) совпадает с \( \varphi \) на \((a, b) \cap (a, t_0 + h)\) по теореме Пикара. Но
            
            \[
            t_0 + h > b - \frac{h}{2} + h = b + \frac{h}{2} > b,
            \]
            
            то есть \( \tilde{\varphi} \) — продолжение \( \varphi \) вправо за точку \( b \). Так как \( \varphi \) по условию является максимальным решением, приходим к противоречию.
            \end{proof}

    \section{Теорема о системе, сравнимой с линейной.}
    \begin{thm}
        Пусть \( G = (a, b) \times \mathbb{R}_r^n \),  
        \( f \in C(G \to \mathbb{R}^n) \cap \text{Lip}_{r, loc} \), функции \( u, v \in C(a, b) \) таковы, что для любой точки  
        \((t, r) \in G\)  
        \[
        |f(t, r)| \leq u(t)|r| + v(t).
        \]
        
        Тогда каждое максимальное решение уравнения \( r' = f(t, r) \) определено на \((a, b)\).
        \end{thm}
        
        \begin{proof}
        По теореме о существовании и единственности максимального решения задачи Коши любая задача Коши с начальными данными \((t_0, r^0) \in G\) имеет единственное максимальное решение \(\varphi\), заданное на некотором интервале \((\alpha, \beta)\). Пусть, например, \(\beta < b\). Применяя равносильное интегральное уравнение (лемма о равносильном интегральном уравнении), при \(t \in [t_0, \beta)\) находим  
        
        \[
        |\varphi(t)| = \left| r^0 + \int_{t_0}^t f(\tau, \varphi(\tau)) \, d\tau \right| \leq |r^0| + \int_{t_0}^t |f(\tau, \varphi(\tau))| \, d\tau \leq |r^0| + \int_{t_0}^t |u(\tau)||\varphi(\tau)| \, d\tau + \int_{t_0}^t |v(\tau)| \, d\tau.
        \]
        
        Из непрерывности функций \(u\) и \(v\) вытекает их ограниченность на отрезке \([t_0, \beta]\). Следовательно, найдутся такие числа \(\lambda, \mu \geq 0\), что при \(t \in [t_0, \beta)\)  
        
        \[
        |\varphi(t)| \leq \lambda + \mu \int_{t_0}^t |\varphi(s)| \, ds.
        \]
        
        Тогда по лемме Гронуолла
        
        \[
        |\varphi(t)| \leq \lambda e^{\mu(t-t_0)} \leq L,
        \]
        
        где \(L = \lambda e^{\mu(\beta-t_0)}\). Отсюда следует, что график решения \(\varphi\) не покидает компакт  
        
        \[
        K = \{(t, r) \in G \mid t \in [t_0, \beta], |r| \leq L\} \subset G
        \]
        
        при \(t \in [t_0, \beta)\), что противоречит теореме о выходе интегральной кривой за пределы компакта.
        \end{proof}
        
    \section{Теорема существования и единственности максимального решения линейной системы (вещественный и комплексный случай).}
    \begin{dfn}
        \textbf{Линейной системой дифференциальных уравнений} называют систему вида

        \[
            r' = P(t)r + q(t).
            \tag{9.1}
        \]
        

    \end{dfn}

    \begin{thm}(Вещественный случай)
        Пусть \( P \in \text{Mat}_n(C((a, b) \to \mathbb{R})), q \in C((a, b) \to \mathbb{R}^n), t_0 \in (a, b), r^0 \in \mathbb{R}^n \). Тогда максимальное решение задачи Коши
        
        \[
        r' = P(t)r + q(t), \quad r(t_0) = r^0
        \tag{9.2}
        \]
        
        существует, единственно и определено на интервале \((a, b)\).
        \end{thm}
        
        \begin{proof}
        Заметим, что правая часть системы \( f(t, r) = P(t)r + q(t) \) и её производная \( f'_r = P(t) \) непрерывны в области \((a, b) \times \mathbb{R}^n\). Тогда по теореме существования и единственности максимального решения
        задачи Коши существует единственное максимальное решение задачи (9.2).
        
        Имеем
        
        \[
        |f(t, r)| \leq |P(t)r| + |q(t)| \leq n|P(t)|\cdot |r| + |q(t)|.
        \]
        
        Так как функции \( u(t) = n|P(t)| \) и \( v(t) = |q(t)| \) непрерывны на \((a, b)\), то по теореме о системе, сравнимой с линейной, решение задачи (9.2) продолжимо на интервал \((a, b)\). \(\square\)
        \end{proof}
        
        \begin{thm}(Комплексный случай)
        Пусть \( P \in \text{Mat}_n(C((a, b))), q \in C((a, b) \to \mathbb{C}^n), t_0 \in (a, b), r^0 \in \mathbb{C}^n \). Тогда максимальное решение задачи Коши
        
        \[
        r' = P(t)r + q(t), \quad r(t_0) = r^0
        \tag{9.3}
        \]
        
        существует, единственно и определено на интервале \((a, b)\).
        \end{thm}
        
        \begin{proof}
        Пусть
        
        \[
        P = A + iB, \quad q = \alpha + i\beta, \quad r = u + iv, \quad r^0 = u_0 + iv_0,
        \]
        
        где \( A, B \in \text{Mat}_n(C((a, b) \to \mathbb{R})), \alpha, \beta, u, v \in C((a, b) \to \mathbb{R}^n), u_0, v_0 \in \mathbb{R}^n \).
        
        \textbf{Единственность.} Пусть \( r \) — максимальное решение задачи (9.3). Тогда
        
        \[
        u' + iv' = (A + iB)(u + iv) + \alpha + i\beta, \quad u(t_0) + iv(t_0) = u_0 + iv_0, \tag{9.4}
        \]
        
        что равносильно
        
        \[
        \begin{bmatrix}
        u' \\
        v'
        \end{bmatrix}
        =
        \begin{bmatrix}
        A & -B \\
        B & A
        \end{bmatrix}
        \begin{bmatrix}
        u \\
        v
        \end{bmatrix}
        +
        \begin{bmatrix}
        \alpha \\
        \beta
        \end{bmatrix},
        \quad \begin{bmatrix}
        u(t_0) \\
        v(t_0)
        \end{bmatrix}
        =
        \begin{bmatrix}
        u_0 \\
        v_0
        \end{bmatrix}.
        \tag{9.5}
        \]
        
        Значит, вектор \((u, v)^T\) — решение задачи (9.5) с вещественными коэффициентами. По вещественному случаю задача (9.5) не может иметь более одного максимального решения. Поэтому, если решение задачи (9.2) существует, то оно единственно.
        
        \textbf{Существование.} По вещественному случаю задача (9.5) имеет максимальное решение \((u, v)^T\), заданное на \((a, b)\). Поскольку соотношения (9.4) и (9.5) равносильны, получаем, что \( r = u + iv \) — решение задачи (9.2) на \((a, b)\). Решение \( r \) непродолжаемо, иначе решение \((u, v)^T\) задачи (9.5) было бы продолжено. \(\square\)
        \end{proof}

    \section{Свойства вронскиана решений линейной однородной системы. Критерий линейной независимости решений линейной однородной системы.}

    \begin{dfn}
        Если \( q \equiv 0 \) на \( (a, b) \), то система (9.1), то есть
        \[
        r' = P(t)r,
        \tag{9.6}
        \]
        называется \textbf{однородной}, в противном случае --- \textbf{неоднородной}.
        \end{dfn}
    \begin{lmm}
        Пусть \((r^k)_{k=1}^n\) — решения системы (9.6). Тогда следующие утверждения равносильны:
        
        (i) \(W \equiv 0\) на \((a, b)\);
        
        (ii) \(W(t_0) = 0\) в некоторой точке \(t_0 \in (a, b)\);
        
        (iii) \((r^k)_{k=1}^n\) линейно зависимы на \((a, b)\).
        \end{lmm}
        
        \begin{proof}
        Проведём доказательство по схеме:  
        (i) \(\Rightarrow\) (ii) \(\Rightarrow\) (iii) \(\Rightarrow\) (i).
        
        \noindent\textbf{(i) \(\Rightarrow\) (ii)} Это следствие очевидно.
        
        \noindent\textbf{(ii) \(\Rightarrow\) (iii)} Пункт (ii) означает, что векторы \((r^k(t_0))_{k=1}^n\) линейно зависимы. Значит, найдётся набор чисел \((c_k)_{k=1}^n\), такой что  
        \[
        \sum_{k=1}^n c_k r^k(t_0) = 0.
        \]
        
        Положим \(\varphi := c_1 r^1 + c_2 r^2 + \dots + c_n r^n\). Тогда \(\varphi\) — решение системы (9.6), удовлетворяющее условию \(\varphi(t_0) = 0\). Но решением этой же задачи Коши является функция, тождественно равная нулю на \((a, b)\). Следовательно, по теореме(существование и единственность максимального решения ЛС с комплексными коэффициентами) будет \(\varphi \equiv 0\) на \((a, b)\). Значит, вектор-функции \((r^k)_{k=1}^n\) линейно зависимы.
        
        \noindent\textbf{(iii) \(\Rightarrow\) (i)} Линейная зависимость вектор-функций \((r^k)_{k=1}^n\) означает линейную зависимость столбцов матрицы \((r^1(t), r^2(t), \dots, r^n(t))\) при любом \(t \in (a, b)\). Тогда её определитель, то есть вронскиан \(W(t)\), тождественно равен нулю. 
        \end{proof}
        
        \begin{thm}
        Пусть \((r^k)_{k=1}^n\) — решения системы (9.6), \(W\) — вронскиан данного набора. Тогда  
        \begin{itemize}
            \item набор \((r^k)_{k=1}^n\) линейно зависим, если и только если \(W(t_0) = 0\) для некоторого \(t_0 \in (a, b)\);  
            \item набор \((r^k)_{k=1}^n\) линейно независим, если и только если \(W(t_0) \neq 0\) для некоторого \(t_0 \in (a, b)\).
        \end{itemize}
    
        \end{thm}
        
        \begin{proof}
        Теорема следует из леммы о свойствах вронскиана.\end{proof}

    \section{Общее решение линейной однородной системы.}

    \begin{thm}
        Пусть \( P \in \text{Mat}_n(C(a, b)) \). Тогда множество решений системы \( r' = P(t)r \) образует n-мерное линейное пространство.
        \end{thm}
        
        \begin{proof}
        Пусть \( t_0 \in (a, b) \), \( (a^k)_{k=1}^n \) — базис в \( \mathbb{C}^n \). По теореме (существование и единственность максимального решения ЛС с комплексными коэффициентами) для любого \( k \in [1:n] \) существует \( r^k \) — решение задачи Коши \( r' = P(t)r \), \( r(t_0) = a^k \). Вронскиан этих решений \( W(t_0) = \det(a^1, a^2, \dots, a^n) \neq 0 \). Тогда по критерию ЛНЗ решений ЛОС функции \( (r^k)_{k=1}^n \) линейно независимы.
        
        Рассмотрим произвольное решение \( r \) системы \( r' = P(t)r \). Пусть \( (c_k)_{k=1}^n \) — координаты вектора \( r(t_0) \) в базисе \( (a^k)_{k=1}^n \). Положим
        
        \[
        \varphi = c_1 r^1 + c_2 r^2 + \dots + c_n r^n.
        \]
        
        Ясно, что \( \varphi \) — решение системы \( r' = P(t)r \), при этом \( \varphi(t_0) = r(t_0) \). Тогда \( r \equiv \varphi \) в силу теоремы о существовании и единственности максимального решения ЛС с комплексными коэффициентами.
        
        Таким образом, функции \( (r^k)_{k=1}^n \) линейно независимы, и любое решение есть их линейная комбинация. Значит, \( (r^k)_{k=1}^n \) — базис в пространстве решений.
        \end{proof}
    
    \section{Лемма о множестве фундаментальных матриц. Лемма об овеществлении фундаментальной системы решений.}

    \begin{lmm}
        Пусть \( \Phi \) — фундаментальная матрица системы (9.6). Тогда
        \[
        \{ \Phi M \mid M \in \operatorname{Mat}_n(\mathbb{C}),\ \det M \neq 0 \}
        \]
        — множество всех фундаментальных матриц этой системы.
    \end{lmm}
        
        \begin{proof}
        Пусть \( \Psi \) — фундаментальная матрица системы (9.6). Тогда каждый её столбец, будучи решением этой системы, является линейной комбинацией столбцов матрицы \( \Phi \). Записывая коэффициенты разложения в столбцы матрицы \( M \), имеем \( \Psi = \Phi M \). А так как \( \det \Psi \neq 0 \) и \( \det \Phi \neq 0 \), то и \( \det M \neq 0 \).
        
        Обратно, пусть \( M \in \operatorname{Mat}_n(\mathbb{C}) \) — произвольная невырожденная матрица. Тогда матрица \( \Phi M \) состоит из решений, а её определитель не обращается в ноль. Следовательно, по критерию ЛНЗ решений ЛОС эти решения линейно независимы, поэтому \( \Phi M \) — фундаментальная матрица.
        \end{proof}
        
        \begin{lmm}
        Пусть \( n \in \mathbb{N} \), \( \Phi = (r^1, r^2, r^3, \ldots, r^n) \) — фундаментальная матрица системы (9.6), при этом \( r^1 = \bar{r}^2 \). Тогда
        \[
        \Psi = (\operatorname{Re} r^1, \operatorname{Im} r^1, r^3, \ldots, r^n)
        \]
        — фундаментальная матрица той же системы.
        \end{lmm}
        
        \begin{proof}
        Так как
        \[
        \operatorname{Re} r^1 = \frac{1}{2}(r^1 + \bar{r}^1) = \frac{1}{2}r^1 + \frac{1}{2}\bar{r}^2, \qquad
        \operatorname{Im} r^1 = \frac{1}{2i}(r^1 - \bar{r}^1) = \frac{1}{2i}r^1 - \frac{1}{2i}\bar{r}^2,
        \]
        то
        \[
        \Psi = \Phi
        \begin{bmatrix}
        \dfrac{1}{2} & \dfrac{1}{2i} & 0\\[6pt]
        \dfrac{1}{2} & -\dfrac{1}{2i} & 0\\[6pt]
        0 & 0 & E_{n-2}
        \end{bmatrix},
        \]
        где \( E_{n-2} \) — единичная матрица порядка \( n-2 \). По лемме о множестве фундаментальных матриц матрица \( \Psi \) является фундаментальной.
        \end{proof}

    \section{Фундаментальная система решений линейной однородной системы с постоянными коэффициентами. Следствие о методе неопределённых коэффициентов.}
    \begin{lmm}
        Пусть $n,k\in\mathbb N$, $A\in\mathrm{Mat}_n(\mathbb C)$, $h^1,h^2,\dots,h^k$ --- жорданова цепочка, соответствующая $\lambda\in\mathrm{spec}\,A$. Тогда вектор-функции
        \[
        \begin{aligned}
        \varphi^1(t)&=e^{\lambda t}h^1,\\
        \varphi^2(t)&=e^{\lambda t}\Big(\frac{t}{1!}h^1+h^2\Big),\\
        &\ \vdots\\
        \varphi^k(t)&=e^{\lambda t}\Big(\frac{t^{k-1}}{(k-1)!}h^1+\frac{t^{k-2}}{(k-2)!}h^2+\dots+\frac{t}{1!}h^{k-1}+h^k\Big)
        \end{aligned}
        \]
        являются решениями системы
        \[
        r'(t)=A r(t).
        \]
        \end{lmm}
        \begin{proof}
            Принимая во внимание определение жордановой цепочки,
            при \( j \in [1 : k] \) имеем
            
            \[
            A \varphi^j = e^{\lambda t} \sum_{m=1}^j \frac{t^{\,j-m}}{(j-m)!} Ah^m = e^{\lambda t} \left( \frac{t^{\,j-1}}{(j-1)!} \lambda h^1 + \sum_{m=2}^j \frac{t^{\,j-m}}{(j-m)!} (\lambda h^m + h^{m-1}) \right) =
            \]
            
            \[
            = e^{\lambda t} \left( \lambda \sum_{m=1}^j \frac{t^{\,j-m}}{(j-m)!} h^m + \sum_{m=2}^j \frac{t^{\,j-m}}{(j-m)!} h^{m-1} \right).
            \]
            
            Это же выражение получается при дифференцировании вектор-функции \( \varphi^j \).
            Значит, \( (\varphi^j)' = A \varphi^j \), что и требовалось.
            \end{proof}
        \begin{thm}
        (ФСР ЛОС с постоянными коэффициентами). Пусть $A\in\mathrm{Mat}_n(\mathbb C)$, и базис пространства $\mathbb C^n$ состоит из жордановых цепочек
        \[
        \lambda_1\sim h^1,h^2,\dots,h^{k},\quad\ldots,\quad
        \lambda_d\sim u^1,u^2,\dots,u^{m},
        \]
        соответствующих $\lambda_1,\dots,\lambda_d\in\mathrm{spec}\,A$. Тогда вектор-функции
        \[
        \begin{aligned}
        \varphi^1(t)&=e^{\lambda_1 t}h^1,\ \dots,\ 
        \varphi^{k}(t)=e^{\lambda_1 t}\Big(\frac{t^{k-1}}{(k-1)!}h^1+\dots+h^{k}\Big),\\
        &\ \vdots\\
        \psi^1(t)&=e^{\lambda_d t}u^1,\ \dots,\ 
        \psi^{m}(t)=e^{\lambda_d t}\Big(\frac{t^{m-1}}{(m-1)!}u^1+\dots+u^{m}\Big)
        \end{aligned}
        \]
        образуют фундаментальную систему решений системы $r'=Ar$.
        \end{thm}
        
        \begin{proof}
        По предыдущей лемме каждое из перечисленных вектор-решений действительно является решением $r'=Ar$. Вронскиан этих вектор-функций в точке $t=0$ равен определителю матрицы, составленной из векторов базиса (жордановых цепочек), поэтому он отличен от нуля:
        \[
        W(0)=\det[h^1,\dots,h^{k},\dots,u^1,\dots,u^{m}]\neq0.
        \]
        Следовательно, функции линейно независимы и образуют фундаментальную систему решений по критерию ЛНЗ решений ЛОС.
        \end{proof}

        \textbf{Следствие.}
        Пусть $\lambda \in \operatorname{spec} A$ имеет алгебраическую кратность $m_a$
        и геометрическую кратность $m_g$.
        Тогда система
        \[
        r' = Ar
        \]
        имеет $m_a$ линейно независимых решений вида
        \begin{equation}
        \varphi(t) = e^{\lambda t} Q^{\,m_a - m_g}(t),
        \tag{10.2}
        \end{equation}
        где $Q^s$ — вектор-многочлен степени не выше $s$.

        \begin{proof}
            По теореме о ФСР ЛОС с постоянными коэффициентами числу $\lambda$ соответствуют $m_g$ групп решений размеров
            $k_1, k_2, \ldots, k_{m_g}$, причём
            \[
            k_1 + k_2 + \ldots + k_{m_g} = m_a.
            \]
            Все эти решения линейно независимы и имеют вид экспоненты,
            умноженной на некоторый вектор-многочлен.
            При этом в $j$-й группе степень многочленов,
            умножаемых на $e^{\lambda t}$, не превосходит $k_j - 1$.
            
            Не умаляя общности, считаем
            \[
            k_1 = \max_{j \in [1 : m_g]} k_j.
            \]
            Тогда степень многочленов не превосходит $k_1 - 1$.
            Так как
            \[
            m_a = k_1 + \ldots + k_{m_g} \ge k_1 + (m_g - 1),
            \]
            то
            \[
            k_1 - 1 \le m_a - m_g,
            \]
            что и требовалось.
        \end{proof}

    \section{Общее решение линейной неоднородной системы.}
    \begin{thm}
    Пусть \( P \in \mathrm{Mat}_n(C(a,b)) \), \( q \in C((a,b)\to \mathbb{C}^n) \),
    \( \varphi \) — решение системы
    \begin{equation}
    r' = P(t)r + q(t), \tag{11.1}
    \end{equation}
    
    \( \Phi \) — фундаментальная матрица системы
    \[
    r' = P(t)r.
    \]
    
    Тогда общее решение неоднородной системы \((11.1)\) имеет вид
    \[
    r = \Phi C + \varphi, \qquad C \in \mathbb{C}^n.
    \]
    \end{thm}
    
    \begin{proof}
    Пусть \( r \) — произвольное решение \((11.1)\). Тогда
    \[
    r' = Pr + q.
    \]
    
    Функция \( \varphi \) удовлетворяет такому же соотношению:
    \[
    \varphi' = P\varphi + q.
    \]
    
    Вычитая эти равенства, находим
    \[
    (r - \varphi)' = P(r - \varphi).
    \]
    
    Значит, найдётся вектор-столбец \( C \in \mathbb{R}^n \), такой что
    \[
    r - \varphi = \Phi C.
    \]
    
    Верно и обратное: любая функция вида \( \Phi C + \varphi \) является решением
    \((11.1)\), что проверяется непосредственной подстановкой. \end{proof}

    \section{Метод вариации постоянных для линейной системы.}
    \begin{thm}
        Пусть $\Phi$ — фундаментальная матрица системы $r' = P(t)r$, $P \in \operatorname{Mat}_n(C(a, b))$, $q \in C((a, b) \to \mathbb{C}^n)$. Тогда общее решение системы (11.1) имеет вид $r = \Phi C$, где вектор-функция $C$ удовлетворяет системе
        \[
        \Phi C' = q.
        \]
    \end{thm}
        
        \begin{proof}
        Опираясь на формулу для обратной матрицы, использующей алгебраические дополнения, заключаем, что $\Phi^{-1} \in \operatorname{Mat}_n(C(a, b))$. Поэтому
        \[
        C(t) = \int \Phi^{-1} q + A,
        \]
        где $A$ — вектор произвольных постоянных. Тогда требуется доказать, что общее решение системы (11.1) имеет вид
        \[
        r = \Phi A + \Phi \int \Phi^{-1} q.
        \]
        По теореме об общем решении ЛНС достаточно показать, что второе слагаемое в правой части — частное решение системы (11.1). Убедимся в этом подстановкой:
        \[
        \Phi' \int \Phi^{-1} q + \Phi \Phi^{-1} q = P(t) \Phi \int \Phi^{-1} q + q.
        \]
        Это верное тождество, поскольку $P(t) \Phi = \Phi'$.
        \end{proof}

    \section{Теоремы о линейных уравнениях n-го порядка: теорема об изоморфизме, критерий линейной независимости решений, общее решение, метод вариации постоянных.}

\begin{dfn} Линейным дифференциальным уравнением порядка $n$ называется уравнение вида  

\[
y^{(n)} + p_{n-1}(t)y^{(n-1)} + \dots + p_1(t)y' + p_0(t)y = q(t), \tag{11.2}
\]

где $p_0, p_1, \dots, p_{n-1}, q \in C(a, b)$. \end{dfn}

\begin{dfn} Если $q \equiv 0$ на $(a, b)$, то уравнение (11.2), то есть  

\[
y^{(n)} + p_{n-1}(t)y^{(n-1)} + \dots + p_1(t)y' + p_0(t)y = 0, \tag{11.3}
\]

называется \textbf{однородным}, в противном случае --- \textbf{неоднородным}. \end{dfn}

\begin{lmm}(О равносильной ЛС) Отображение $\Lambda_n \varphi = [\varphi, \varphi', \dots, \varphi^{(n-1)}]^T$ --- биекция между решениями уравнения (11.2) и решениями системы  

\[
r' = P(t)r + Q(t), \tag{11.4}
\]

где  

\[
P = 
\begin{bmatrix}
0 & 1 & 0 & \dots & 0 \\
0 & 0 & 1 & \dots & 0 \\
\vdots & \vdots & \vdots & \ddots & \vdots \\
0 & 0 & 0 & \dots & 1 \\
-p_0 & -p_1 & -p_2 & \dots & -p_{n-1}
\end{bmatrix}, \quad
Q = 
\begin{bmatrix}
0 \\
0 \\
\vdots \\
0 \\
q
\end{bmatrix}.
\]
\end{lmm}
\begin{proof} В силу леммы о системы, равносильной уравнению. \end{proof}

    \begin{thm}(Об изоморфизме) Пусть \( p_0, p_1, \ldots, p_{n-1} \in C(a, b) \). Тогда множество решений однородного уравнения (11.3) является линейным пространством, изоморфным пространству решений системы

\[
r' = P(t)r, \tag{11.6}
\]

где матрица \( P \) та же, что и в (11.4). При этом изоморфизм устанавливается отображением \( \Lambda_n \). \end{thm}

\begin{proof} Любое решение уравнения (11.3) является элементом линейного пространства \( C(a, b) \). Кроме того, сумма двух решений, а также решение, умноженное на произвольное число, также являются решениями. Поэтому множество всех решений само образует линейное пространство.

По лемме о равносильной ЛС отображение \( \Lambda_n \) устанавливает биекцию между решениями уравнения и равносильной системы. Отображение \( \Lambda_n \) линейно. Таким образом, \( \Lambda_n \) — изоморфизм.
\end{proof}

\begin{lmm}(Cвойства вронскиана решений ЛОУ)
    Пусть $(y_k)_{k=1}^n$ — решения уравнения (11.3). Тогда следующие утверждения равносильны:
    
    \begin{enumerate}
        \item[(i)] $w \equiv 0$ на $(a, b)$;
        \item[(ii)] $w(t_0) = 0$ в некоторой точке $t_0 \in (a, b)$;
        \item[(iii)] $(y_k)_{k=1}^n$ линейно зависимы на $(a, b)$.
    \end{enumerate}\end{lmm}
    
\begin{proof} Так как вронскиан функций $(y_k)_{k=1}^n$ совпадает с вронскианом вектор-функций $(\Lambda_n y_k)_{k=1}^n$, то для доказательства достаточно принять во внимание критерий ЛНЗ решений ЛОС.
    
    \hfill \end{proof}
    
    \begin{lmm}(Критерий линейной независимости решений ЛОУ)
    Пусть $(y_k)_{k=1}^n$ — решения уравнения (11.3), $w$ — вронскиан этого набора. Тогда
    
    \begin{itemize}
        \item набор $(y_k)_{k=1}^n$ линейно зависим, если и только если $w(t_0) = 0$ для некоторого $t_0 \in (a, b)$;
        \item набор $(y_k)_{k=1}^n$ линейно независим, если и только если $w(t_0) \neq 0$ для некоторого $t_0 \in (a, b)$.
    \end{itemize}
\end{lmm}
    
    \begin{proof}Следует из леммы о свйоствах вронскиана решений ЛОУ.
    
    \hfill \end{proof}

    \begin{thm}(Общее решение ЛУ)
        Пусть \( p_0, p_1, \dots, p_{n-1}, q \in C(a, b) \), \(\psi\) — решение уравнения (11.2), \((\varphi_k)_{k=1}^n\) — фундаментальная система решений уравнения (11.3). Тогда общее решение уравнения (11.2) имеет вид
        \[
        y = \sum_{k=1}^n C_k \varphi_k + \psi,
        \]
        где \((C_k)_{k=1}^n\) — произвольные постоянные.
        \end{thm}
        
        \begin{proof}
        По теореме об изоморфизме вектор-функции \((\Lambda_n \varphi_k)_{k=1}^n\) образуют базис в пространстве решений однородной системы (11.6). В силу леммы о равносильной ЛС вектор-функция \(\Lambda_n \psi\) — решение неоднородной системы (11.4). По теореме об общем решении ЛС
        \[
        r = \sum_{k=1}^n C_k \Lambda_n \varphi_k + \Lambda_n \psi
        \]
        — общее решение системы (11.4). По лемме о равносильной ЛС первая компонента правой части даёт формулу общего решения уравнения (11.2).
        \end{proof}
        \begin{thm}(Метод вариации постоянных для ЛУ)
            Пусть \((\varphi_k)_{k=1}^n\) — фундаментальная система решений однородного уравнения (11.3). Тогда общее решение уравнения (11.2) имеет вид \(y = \sum_{k=1}^n C_k \varphi_k\), где функции \((C_k)_{k=1}^n\) пробегают все решения системы
            \[
            \begin{bmatrix}
            \varphi_1 & \varphi_2 & \cdots & \varphi_n \\
            \varphi'_1 & \varphi'_2 & \cdots & \varphi'_n \\
            \vdots & \vdots & \ddots & \vdots \\
            \varphi^{(n-2)}_1 & \varphi^{(n-2)}_2 & \cdots & \varphi^{(n-2)}_n \\
            \varphi^{(n-1)}_1 & \varphi^{(n-1)}_2 & \cdots & \varphi^{(n-1)}_n
            \end{bmatrix}
            \begin{bmatrix}
            C'_1 \\
            C'_2 \\
            \vdots \\
            C'_{n-1} \\
            C'_n
            \end{bmatrix}
            =
            \begin{bmatrix}
            0 \\
            0 \\
            \vdots \\
            0 \\
            q
            \end{bmatrix}.
            \]
            \end{thm}
            
            \begin{proof}
            По теореме о методе вариации постоянных для ЛС общее решение системы, равносильной уравнению (11.2), имеет вид
            \[
            r = \sum_{k=1}^n C_k \Lambda \varphi_k,
            \]
            где функции \(C_1, \ldots, C_n\) удовлетворяют системе
            \[
            (\Lambda \varphi_1, \ldots, \Lambda \varphi_n)
            \begin{bmatrix}
            C'_1 \\
            \vdots \\
            C'_{n-1} \\
            C'_n
            \end{bmatrix}
            =
            \begin{bmatrix}
            0 \\
            \vdots \\
            0 \\
            q
            \end{bmatrix}.
            \]
            По лемме о равносильной ЛС первая строка вектора \(r\) — общее решение уравнения (11.2).
            \end{proof}
\end{document}